%%%%%%%%%%%%%%%%%%%%%%%%%%%%%%%%%%%%%%%%%%%%%%%%%%%%%%%%%%%%%%%%%%%%%%%%%%%%
% AGUtmpl.tex: this template file is for articles formatted with LaTeX2e,
% Modified December 2018
%
% This template includes commands and instructions
% given in the order necessary to produce a final output that will
% satisfy AGU requirements.
%
% FOR FIGURES, DO NOT USE \psfrag
%
%%%%%%%%%%%%%%%%%%%%%%%%%%%%%%%%%%%%%%%%%%%%%%%%%%%%%%%%%%%%%%%%%%%%%%%%%%%%
%
% IMPORTANT NOTE:
%
% SUPPORTING INFORMATION DOCUMENTATION IS NOT COPYEDITED BEFORE PUBLICATION.
%
%
%
%%%%%%%%%%%%%%%%%%%%%%%%%%%%%%%%%%%%%%%%%%%%%%%%%%%%%%%%%%%%%%%%%%%%%%%%%%%%
%
% Step 1: Set the \documentclass
%
%
% PLEASE USE THE DRAFT OPTION TO SUBMIT YOUR PAPERS.
% The draft option produces double spaced output.
%
% Choose the journal abbreviation for the journal you are
% submitting to:

% jgrga JOURNAL OF GEOPHYSICAL RESEARCH (use for all of them)
% gbc   GLOBAL BIOCHEMICAL CYCLES
% grl   GEOPHYSICAL RESEARCH LETTERS
% pal   PALEOCEANOGRAPHY
% ras   RADIO SCIENCE
% rog   REVIEWS OF GEOPHYSICS
% tec   TECTONICS
% wrr   WATER RESOURCES RESEARCH
% gc    GEOCHEMISTRY, GEOPHYSICS, GEOSYSTEMS
% sw    SPACE WEATHER
% ms    JAMES
% ef    EARTH'S FUTURE
%
%
%
% (If you are submitting to a journal other than jgrga,
% substitute the initials of the journal for "jgrga" below.)

\documentclass[draft,jgrga]{agutexSI2019}

%%%%%%%%%%%%%%%%%%%%%%%%%%%%%%%%%%%%%%%%%%%%%%%%%%%%%%%%%%%%%%%%%%%%%%%%%
%
%  SUPPORTING INFORMATION TEMPLATE
%
%% ------------------------------------------------------------------------ %%
%
%
%Please use this template when formatting and submitting your Supporting Information.

%This template serves as both a “table of contents” for the supporting information for your article and as a summary of files.
%
%
%OVERVIEW
%
%Please note that all supporting information will be peer reviewed with your manuscript. It will not be copyedited if the paper is accepted.
%In general, the purpose of the supporting information is to enable authors to provide and archive auxiliary information such as data tables, method information, figures, video, or computer software, in digital formats so that other scientists can use it.
%The key criteria are that the data:
% 1. supplement the main scientific conclusions of the paper but are not essential to the conclusions (with the exception of
%    including %data so the experiment can be reproducible);
% 2. are likely to be usable or used by other scientists working in the field;
% 3. are described with sufficient precision that other scientists can understand them, and
% 4. are not exe files.
%
%USING THIS TEMPLATE
%
%***All references should be included in the reference list of the main paper so that they can be indexed, linked, and counted as citations.  The reference section does not count toward length limits.
%
%All Supporting text and figures should be included in this document. Insert supporting information content into each appropriate section of the template. To add additional captions, simply copy and paste each sample as needed.

%Tables may be included, but can also be uploaded separately, especially if they are larger than 1 page, or if necessary for retaining table formatting. Data sets, large tables, movie files, and audio files should be uploaded separately. Include their captions in this document and list the file name with the caption. You will be prompted to upload these files on the Upload Files tab during the submission process, using file type “Supporting Information (SI)”

%IMPORTANT NOTE ON FIGURES AND TABLES
% Placeholders for figures and tables appear after the \end{article} command, after references.
% DO NOT USE \psfrag or \subfigure commands.
%
 \usepackage{graphicx}
 \usepackage{color}
%
%  Uncomment the following command to allow illustrations to print
%   when using Draft:
 \setkeys{Gin}{draft=false}
%
% You may need to use one of these options for graphicx depending on the driver program you are using. 
%
% [xdvi], [dvipdf], [dvipsone], [dviwindo], [emtex], [dviwin],
% [pctexps],  [pctexwin],  [pctexhp],  [pctex32], [truetex], [tcidvi],
% [oztex], [textures]
%
%
%% ------------------------------------------------------------------------ %%
%
%  ENTER PREAMBLE
%
%% ------------------------------------------------------------------------ %%

% Author names in capital letters:
%\authorrunninghead{BALES ET AL.}

% Shorter version of title entered in capital letters:
%\titlerunninghead{SHORT TITLE}

%Corresponding author mailing address and e-mail address:
%\authoraddr{Corresponding author: A. B. Smith,
%Department of Hydrology and Water Resources, University of
%Arizona, Harshbarger Building 11, Tucson, AZ 85721, USA.
%(a.b.smith@hwr.arizona.edu)}

\begin{document}

%% ------------------------------------------------------------------------ %%
%
%  TITLE
%
%% ------------------------------------------------------------------------ %%

%\includegraphics{agu_pubart-white_reduced.eps}


\title{Supporting Information for Microburst Size Distribution Derived with AeroCube-6}

%% ------------------------------------------------------------------------ %%
%
%  AUTHORS AND AFFILIATIONS
%
%% ------------------------------------------------------------------------ %%

\authors{M. Shumko\affil{1}, A. Johnson\affil{1}, J. Sample\affil{1}, B.A. Griffith\affil{1}, D.L. Turner\affil{2}, T.P. O'Brien\affil{2},  J.B. Blake\affil{2}, O. Agapitov\affil{3}, S. G. Claudepierre\affil{2}}


\affiliation{1}{Department of Physics, Montana State University, Bozeman, Montana, USA}
\affiliation{2}{Space Science Applications Laboratory, The Aerospace Corportation, El Segundo, California, USA}
\affiliation{3}{Space Sciences Laboratory, University of California berkeley, Berkeley, California, USA}


\begin{article}

\noindent\textbf{Contents of this file}
%%%Remove or add items as needed%%%
\begin{enumerate}
\item Introduction
\item Text S1: Analytic Derivation of $\bar{F}(s)$
\item Text S2: Most probable parameter values for continuous microburst PDFs
\item Text S3: Comparison of microburst to whistler mode chorus $\bar{F}(s)$
%if Tables are larger than 1 page, upload as separate excel file
\end{enumerate}

\noindent\textbf{Introduction}
This supporting information document contains texts S1-S3. Text S1 derives the analytic model that transforms a prescribed microburst PDF into a $\bar{F}$ curve as a function of AC6 separation, $s$. Text S2 expands on the two-sized microburst model results presented in Section 5.3 and the range of optimal model parameters assuming continuous microburst PDFs such as the log-normal, Weibull, and Maxwellian. Lastly, text S3 presents the percent of microbursts observed in each separation bin, as a function of separation and compares is to the observed scale size of chorus waves as a function of wave amplitude.

\clearpage

%Delete all unused file types below. Copy/paste for multiples of each file type as needed.
\noindent\textbf{Text S1: Analytic Derivation of $\bar{F}(s)$}
Here we derive the integral form of $\bar{F}(s)$ under the following assumptions:

\begin{enumerate}
\item microbursts are circular with radius $r$
\item microbursts are randomly and uniformly distributed around AC6.
\end{enumerate}

First recall the area $A(r, s)$, given in Eq. 4 in the main text and copied here for convenience
\begin{equation}
A(r, s) = 2r^2 \cos^{-1}{\Big( \frac{s}{2r} \Big)} - \frac{s}{2} \sqrt{4r^2 - s^2}.
\end{equation} A circular microburst who's center lies in $A(r, s)$ will be observed by both AC6 units and is counted in $\bar{F}(s)$. Now we derive the integral form of $\bar{F}(s)$ that accounts for the different spacecraft separations and microburst sizes that are distributed by a hypothesized PDF $p(r, \theta)$.

First we will account for the effects of various spacecraft separation, assuming all microbursts are one size. For reference choose of radius, $r_0$ and spacecraft separation, $s_0$ such that $A(r_0, s_0) > 0$ which implies that some number of microbursts, $n_0$ will be simultaneously observed. Now, if for example the spacecraft separation (or microburst radius) is changed such that the area doubles, the second assumption implies that the number of microbursts observed during the same time interval must double as well. This can be expressed as 

\begin{equation} \label{density_eq}
\frac{n_0}{A(r_0, s_0)} = \frac{n}{A(r, s)}
\end{equation} and interpreted as the conservation of the microburst area density. By rewriting Eq. \ref{density_eq} as

\begin{equation}
n(r, s) = \bigg( \frac{n_0}{A(r_0, s_0)} \bigg) A(r, s)
\end{equation} it is more clear that the number of microbursts of size $r$ observed at separation $s$ is just $A(r, s)$ scaled by the reference microburst area density. The cumulative number of microbursts observed above $s$ is then

\begin{equation}
N(r, s) = \int_{s}^\infty n(r, s') ds' = \bigg( \frac{n_0}{A(r_0, s_0)} \bigg) \int_{s}^\infty A(r, s') ds'.
\end{equation} Lastly, $\bar{F}(s)$ for a single $r$ is then

\begin{equation}
\bar{F}(s) = \frac{N(s)}{N(0)} = \frac{\int_{s}^\infty A(r, s') ds'}{\int_{0}^\infty A(r, s') ds'}
\end{equation}

To incorporate a continuous microburst PDF such as $p(r) = p_1 \delta (r-r_1) + p_2 \delta (r-r_2) + ...$ we sum up the weighted number of microbursts that each size contributes to $N(s)$ i.e.

\begin{equation}
N(s) = \bigg( \frac{n_0}{A(r_0, s_0)} \bigg) \bigg( \int_{s}^\infty p_1 A(r_1, s') ds' + \int_{s}^\infty p_2 A(r_2, s') ds' + ...\bigg)
\end{equation}

The last step is to convert the sum of Dirac Delta functions into a continuous PDF $p(r)$ after which 

\begin{equation}
N(s) = \bigg( \frac{n_0}{A(r_0, s_0)} \bigg) \displaystyle\int\displaylimits_{s}^{\infty} \displaystyle\int\displaylimits_0^{\infty} A(r, s') p(r) dr ds'.
\end{equation} With these considerations, $\bar{F}(s)$ is then given by 

\begin{equation} \label{analytic_integral}
\bar{F}(s, \theta) = \frac{\displaystyle\int\displaylimits_{s}^{\infty} \displaystyle\int\displaylimits_0^{\infty} A(r, s') p(r, \theta) dr ds'}{\displaystyle\int\displaylimits_{0}^{\infty} \displaystyle\int\displaylimits_0^{\infty} A(r, s') p(r, \theta) dr ds'}
\end{equation}
\clearpage 

\noindent\textbf{Text S2: Most probable parameter values for continuous microburst PDFs}

Besides the one and two-size microburst models described in the main text, continuous PDFs such as the log-normal, Weibull, and Maxwellian were fit and their optimal parameters presented here.

For the Maxwellian PDF, we assumed the following form

\begin{equation}
p(r | a) = \sqrt{\frac{2}{\pi}} \frac{r^2 e^{-r^2/(2a^2)}}{a^3}.
\end{equation} The range of $a$ consistent with the observed data was found to be between 0 and 35 km. Next, the log-normal distribution of the following form was used
\begin{equation}
p(r | \mu, \sigma) = \frac{1}{\sigma r \sqrt{2 \pi}} e^{\Big( -\big( ln(r) - ln(\mu) \big)^2/(2 \sigma^2) \Big)}
\end{equation} and the results are summarized in \ref{table_s1}. Lastly the Weibull distribution of the following form was tested
\begin{equation}
p(r | c, r_0, \lambda) = c \bigg(\frac{r-r_0}{\lambda}\bigg)^{c-1} exp \Bigg(- \bigg(\frac{r-r_0}{\lambda}\bigg)^{c} \Bigg).
\end{equation} for which the model parameters are summarized in Table \ref{table_s2}.

\begin{table}[h]
\settablenum{S1} %%Change number for each table
\caption{Range of log-normal model parameters consistent with the observed AC6  $\bar{F}(s)$}
\label{table_s1}
\centering
\begin{tabular}{|c|c|c|}
\hline 
percentile (\%) & $\mu$ & $\sigma$ \\ 
\hline 
2.5 & 1.8 & 0 \\ 
\hline 
50 & 21.8 & 0.4 \\ 
\hline 
97.5 & 52.0 & 1.1 \\ 
\hline 
\end{tabular} 
\end{table}

\begin{table}[h]
\settablenum{S2} %%Change number for each table
\caption{Range of Weibull model parameters consistent with the observed AC6  $\bar{F}(s)$}
\label{table_s2}
\centering
\begin{tabular}{|c|c|c|c|}
\hline 
percentile (\%) & c & $r_0$ & $\lambda$ \\ 
\hline 
2.5 & 0.6 & 1.3 & 2.7 \\ 
\hline 
50 & 5.5 & 26.2 & 32 \\ 
\hline 
97.5 & 19.3 & 72.5 & 72.2 \\ 
\hline 
\end{tabular} 
\end{table}

\clearpage 
\noindent\textbf{Text S3: Comparison of microburst to whistler mode chorus $\bar{F}(s)$}

\textcolor{red}{TBD}

%% ------------------------------------------------------------------------ %%
%%  REFERENCE LIST AND TEXT CITATIONS

%%%%%%%%%%%%%%%%%%%%%%%%%%%%%%%%%%%%%%%%%%%%%%%
% 
%
% \bibliography{<name of your .bib file>} do not specify file extension
%
% no need to specify bibliographystyle
%
% Note that ALL references in this supporting information file must also be referenced in the primary manuscript
%
%%%%%%%%%%%%%%%%%%%%%%%%%%%%%%%%%%%%%%%%%%%%%%%
% if you get an error about newblock being undefined, uncomment this line:
%\newcommand{\newblock}{}
%\bibliography{<name of your .bib file>} 




%Reference citation instructions and examples:
%
% Please use ONLY \cite and \citeA for reference citations.
% \cite for parenthetical references
% ...as shown in recent studies (Simpson et al., 2019)
% \citeA for in-text citations
% ...Simpson et al (2019) have shown...
% DO NOT use other cite commands (e.g., \citet, \citep, \citeyear, \nocite, \citealp, etc.).
%
%
%...as shown by \citeA{jskilby}.
%...as shown by \citeA{lewin76}, \citeA{carson86}, \citeA{bartoldy02}, and \citeA{rinaldi03}.
%...has been shown \cite{jskilbye}.
%...has been shown \cite{lewin76,carson86,bartoldy02,rinaldi03}.
%...has been shown \cite{lewin76,carson86,bartoldy02,rinaldi03}.
%
% DO NOT use other cite commands (e.g., \citet, \citep, \citeyear, \nocite, \citealp, etc.).
%

%% ------------------------------------------------------------------------ %%
%
%  END ARTICLE
%
%% ------------------------------------------------------------------------ %%
\end{article}
\clearpage

% Copy/paste for multiples of each file type as needed.

% enter figures and tables below here: %%%%%%%
%
%
%
%
% EXAMPLE FIGURES
% ---------------
% If you get an error about an unknown bounding box, try specifying the width and height of the figure with the natwidth and natheight options.
% \begin{figure}
%\setfigurenum{S1} %%You can change number for each figure if you want, not required. "S" prepended automatically.
% \noindent\includegraphics[natwidth=800px,natheight=600px]{samplefigure.eps}
%\caption{caption}
%\label{epsfiguresample}
%\end{figure}
%
%
% Giving latex a width will help it to scale the figure properly. A simple trick is to use \textwidth. Try this if large figures run off the side of the page.
% \begin{figure}
% \noindent\includegraphics[width=\textwidth]{anothersample.png}
%\caption{caption}
%\label{pngfiguresample}
%\end{figure}
%
%
%\begin{figure}
%\noindent\includegraphics[width=\textwidth]{athirdsample.pdf}
%\caption{A pdf test figure}
%\label{pdffiguresample}
%\end{figure}
%
% PDFLatex does not seem to be able to process EPS figures. You may want to try the epstopdf package.
%
%
% ---------------
% EXAMPLE TABLE
%
%\begin{table}
%\settablenum{S1} %%Change number for each table
%\caption{Time of the Transition Between Phase 1 and Phase 2\tablenotemark{a}}
%\centering
%\begin{tabular}{l c}
%\hline
% Run  & Time (min)  \\
%\hline
%  $l1$  & 260   \\
%  $l2$  & 300   \\
%  $l3$  & 340   \\
%  $h1$  & 270   \\
%  $h2$  & 250   \\
%  $h3$  & 380   \\
%  $r1$  & 370   \\
%  $r2$  & 390   \\
%\hline
%\end{tabular}
%\tablenotetext{a}{Footnote text here.}
%\end{table}
% ---------------
%
% EXAMPLE LARGE TABLE (UPLOADED SEPARATELY)
%\begin{table}
%\settablenum{S1} %%Change number for each table
%\caption{Time of the Transition Between Phase 1 and Phase 2\tablenotemark{a}}
%\end{table}


\end{document}

%%%%%%%%%%%%%%%%%%%%%%%%%%%%%%%%%%%%%%%%%%%%%%%%%%%%%%%%%%%%%%%

More Information and Advice:

%% ------------------------------------------------------------------------ %%
%
%  SECTION HEADS
%
%% ------------------------------------------------------------------------ %%

% Capitalize the first letter of each word (except for
% prepositions, conjunctions, and articles that are
% three or fewer letters).

% AGU follows standard outline style; therefore, there cannot be a section 1 without
% a section 2, or a section 2.3.1 without a section 2.3.2.
% Please make sure your section numbers are balanced.
% ---------------
% Level 1 head
%
% Use the \section{} command to identify level 1 heads;
% type the appropriate head wording between the curly
% brackets, as shown below.
%
%An example:
%\section{Level 1 Head: Introduction}
%
% ---------------
% Level 2 head
%
% Use the \subsection{} command to identify level 2 heads.
%An example:
%\subsection{Level 2 Head}
%
% ---------------
% Level 3 head
%
% Use the \subsubsection{} command to identify level 3 heads
%An example:
%\subsubsection{Level 3 Head}
%
%---------------
% Level 4 head
%
% Use the \subsubsubsection{} command to identify level 3 heads
% An example:
%\subsubsubsection{Level 4 Head} An example.
%
%% ------------------------------------------------------------------------ %%
%
%  IN-TEXT LISTS
%
%% ------------------------------------------------------------------------ %%
%
% Do not use bulleted lists; enumerated lists are okay.
% \begin{enumerate}
% \item
% \item
% \item
% \end{enumerate}
%
%% ------------------------------------------------------------------------ %%
%
%  EQUATIONS
%
%% ------------------------------------------------------------------------ %%

% Single-line equations are centered.
% Equation arrays will appear left-aligned.

Math coded inside display math mode \[ ...\]
 will not be numbered, e.g.,:
 \[ x^2=y^2 + z^2\]

 Math coded inside \begin{equation} and \end{equation} will
 be automatically numbered, e.g.,:
 \begin{equation}
 x^2=y^2 + z^2
 \end{equation}

% IF YOU HAVE MULTI-LINE EQUATIONS, PLEASE
% BREAK THE EQUATIONS INTO TWO OR MORE LINES
% OF SINGLE COLUMN WIDTH (20 pc, 8.3 cm)
% using double backslashes (\\).

% To create multiline equations, use the
% \begin{eqnarray} and \end{eqnarray} environment
% as demonstrated below.
\begin{eqnarray}
  x_{1} & = & (x - x_{0}) \cos \Theta \nonumber \\
        && + (y - y_{0}) \sin \Theta  \nonumber \\
  y_{1} & = & -(x - x_{0}) \sin \Theta \nonumber \\
        && + (y - y_{0}) \cos \Theta.
\end{eqnarray}

%If you don't want an equation number, use the star form:
%\begin{eqnarray*}...\end{eqnarray*}

% Break each line at a sign of operation
% (+, -, etc.) if possible, with the sign of operation
% on the new line.

% Indent second and subsequent lines to align with
% the first character following the equal sign on the
% first line.

% Use an \hspace{} command to insert horizontal space
% into your equation if necessary. Place an appropriate
% unit of measure between the curly braces, e.g.
% \hspace{1in}; you may have to experiment to achieve
% the correct amount of space.


%% ------------------------------------------------------------------------ %%
%
%  EQUATION NUMBERING: COUNTER
%
%% ------------------------------------------------------------------------ %%

% You may change equation numbering by resetting
% the equation counter or by explicitly numbering
% an equation.

% To explicitly number an equation, type \eqnum{}
% (with the desired number between the brackets)
% after the \begin{equation} or \begin{eqnarray}
% command.  The \eqnum{} command will affect only
% the equation it appears with; LaTeX will number
% any equations appearing later in the manuscript
% according to the equation counter.
%

% If you have a multiline equation that needs only
% one equation number, use a \nonumber command in
% front of the double backslashes (\\) as shown in
% the multiline equation above.

%% ------------------------------------------------------------------------ %%
%
%  SIDEWAYS FIGURE AND TABLE EXAMPLES
%
%% ------------------------------------------------------------------------ %%
%
% For tables and figures, add \usepackage{rotating} to the paper and add the rotating.sty file to the folder.
% AGU prefers the use of {sidewaystable} over {landscapetable} as it causes fewer problems.
%
% \begin{sidewaysfigure}
% \includegraphics[width=20pc]{samplefigure.eps}
% \caption{caption here}
% \label{label_here}
% \end{sidewaysfigure}
%
%
%
% \begin{sidewaystable}
% \caption{}
% \begin{tabular}
% Table layout here.
% \end{tabular}
% \end{sidewaystable}
%
%

