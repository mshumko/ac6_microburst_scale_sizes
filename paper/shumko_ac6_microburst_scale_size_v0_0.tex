%%%%%%%%%%%%%%%%%%%%%%%%%%%%%%%%%%%%%%%%%%%%%%%%%%%%%%%%%%%%%%%%%%%%%%%%%%%%
% AGUJournalTemplate.tex: this template file is for articles formatted with LaTeX
%
% This file includes commands and instructions
% given in the order necessary to produce a final output that will
% satisfy AGU requirements, including customized APA reference formatting.
%
% You may copy this file and give it your
% article name, and enter your text.
%
%
% Step 1: Set the \documentclass
%
% There are two options for article format:
%
% PLEASE USE THE DRAFT OPTION TO SUBMIT YOUR PAPERS.
% The draft option produces double spaced output.
%

%% To submit your paper:
\documentclass[draft]{agujournal2019}
\usepackage{url} %this package should fix any errors with URLs in refs.
\usepackage{lineno}
\usepackage{color}
\graphicspath{ {figures/} }
\linenumbers
%%%%%%%
% As of 2018 we recommend use of the TrackChanges package to mark revisions.
% The trackchanges package adds five new LaTeX commands:
%
%  \note[editor]{The note}
%  \annote[editor]{Text to annotate}{The note}
%  \add[editor]{Text to add}
%  \remove[editor]{Text to remove}
%  \change[editor]{Text to remove}{Text to add}
%
% complete documentation is here: http://trackchanges.sourceforge.net/
%%%%%%%

\draftfalse

%% Enter journal name below.
%% Choose from this list of Journals:
%
% JGR: Atmospheres
% JGR: Biogeosciences
% JGR: Earth Surface
% JGR: Oceans
% JGR: Planets
% JGR: Solid Earth
% JGR: Space Physics
% Global Biogeochemical Cycles
% Geophysical Research Letters
% Paleoceanography and Paleoclimatology
% Radio Science
% Reviews of Geophysics
% Tectonics
% Space Weather
% Water Resources Research
% Geochemistry, Geophysics, Geosystems
% Journal of Advances in Modeling Earth Systems (JAMES)
% Earth's Future
% Earth and Space Science
% Geohealth
%
% ie, \journalname{Water Resources Research}

\journalname{Geophysical Research Letters}


\begin{document}

\title{Microburst Scale Size Distribution Derived with AeroCube-6}

%% ------------------------------------------------------------------------ %%
%
%  AUTHORS AND AFFILIATIONS
%
%% ------------------------------------------------------------------------ %%

\authors{M. Shumko\affil{1}, T.P. O'Brien\affil{2}, J. Sample\affil{1}, A. Johnson\affil{1}, D.L. Turner\affil{2}, J.B. Blake\affil{2}, B.A. Griffith\affil{1}, S. Claudepierre\affil{2}, O. Agapitov\affil{3}}


\affiliation{1}{Department of Physics, Montana State University, Bozeman, Montana, USA}
\affiliation{2}{Space Science Applications Laboratory, The Aerospace Corportation, El Segundo, California, USA}
\affiliation{3}{Space Sciences Laboratory, University of California berkeley, Berkeley, California, USA}

\correspondingauthor{M. Shumko}{msshumko@gmail.com}

%% Keypoints, final entry on title page.
%  List up to three key points (at least one is required)
%  Key Points summarize the main points and conclusions of the article
%  Each must be 100 characters or less with no special characters or punctuation and must be complete sentences

\begin{keypoints}
\item Microburst scale size in low Earth orbit and the magnetic equator was estimated.
\item Majority of microbursts in low Earth orbit have a scale size on the order of 10 km.
\item The majority of microbursts correspond to the correlation scale of \textcolor{red}{high amplitude?} whistler-mode chorus waves at the magnetic equator.
\end{keypoints}

%% ------------------------------------------------------------------------ %%
%
%  ABSTRACT
%
% A good abstract will begin with a short description of the problem
% being addressed, briefly describe the new data or analyses, then
% briefly states the main conclusion(s) and how they are supported and
% uncertainties.
%% ------------------------------------------------------------------------ %%

%% \begin{abstract} starts the second page

\begin{abstract}
enter abstract here

\end{abstract}

\section{Plain Language Summary}
https://sharingscience.agu.org/creating-plain-language-summary/



%% ------------------------------------------------------------------------ %%
%
%  TEXT
%
%% ------------------------------------------------------------------------ %%

%%% Suggested section heads:
% \section{Introduction}
%
% The main text should start with an introduction. Except for short
% manuscripts (such as comments and replies), the text should be divided
% into sections, each with its own heading.

% Headings should be sentence fragments and do not begin with a
% lowercase letter or number. Examples of good headings are:

% \section{Materials and Methods}
% Here is text on Materials and Methods.
%
% \subsection{A descriptive heading about methods}
% More about Methods.
%
% \section{Data} (Or section title might be a descriptive heading about data)
%
% \section{Results} (Or section title might be a descriptive heading about the
% results)
%
% \section{Conclusions}


\section{Introduction}
Since the discovery of the Van Allen radiation belts in the 1960s, decades of research has made headway in understanding the dynamics of particle acceleration and loss mechanisms. One of these mechanisms is wave-particle scattering between whistler-mode chorus waves and electrons which has been modeled and observed as a source of electron acceleration and loss. DESCRIBE CHORUS WAVES AND THEIR GENERATION MECHANISM. Whistler mode chorus is widely believed to cause electron precipitation termed microbursts. Microbursts are a subsecond impulse increase of electrons that were first observed by high altitude balloons and satellites in low Earth orbit (LEO). Microburst’s role as a radiation belt electron loss mechanism has been estimated to be significant, with total radiation belt electron depletion due to microbursts estimated to be on the order of a day. 

One of the unknown characteristics of microbursts that is critical to better quantify the role of microbursts as a loss mechanism is their size. Microburst size, together with their occurrence frequency (anything else?) are necessary parameters to more accurately quantify their contribution to radiation belt electron losses. Furthermore, by comparing the microburst scale size distribution at the magnetic equator to the wave scale sizes estimated in prior literature, the dominant scattering mechanism can be identified. Prior case studies have estimated the LEO microburst scale size between 4 and 50 km, an order of magnitude difference (cite Bern, me, Parks, Sarah, Alex). The large variance in prior results imply that there is a distribution of microburst scale sizes that this study aims to estimate.

This study estimates the microburst scale size distributions in LEO and the magnetic equator and compares it to the scale size of the progenitor waves. The twin AeroCube-6 (AC6) CubeSats which took data together for three years with varying spacecraft separation between 2 and 800 km are utilized for this study. We first introduce the AC-6 mission including their orbit and instrumentation. Then we describe the procedure undertaken to identify microbursts observed by each spacecraft and how they are combined to make a list of the temporally coincident microbursts. Next, the procedure used to estimate the microburst scale size distributions in LEO and the magnetic equator is explained. Lastly, we summarize and compare these results to the microburst scale sizes estimated in prior literature and infer the properties of the whistler-mode chorus waves that are believed to cause microbursts.

\section{Instrumentation}
The AC6 mission consists of a pair of 0.5U (10x10x5 cm) CubeSats built by the Aerospace Corporation and launched on June 19th, 2014 into a 620 x 700 km, 98 degree inclination orbit. The two satellites, designated as AC6-A and AC6-B separated after launch and drifted apart. AC6 has an active attitude control system which allows them to change their differential drag to allow fine separation control. Figure 1a shows the AC6 separation for the duration of the mission.

Each AC6 unit is equipped with a three Aerospace microdosimeters (licensed to Teledyne Microelectronics, Inc). The dosimeter used for this study is dos1 and is identical on both AC6 units. Dos1 has a 30 keV electron threshold and samples at 10 Hz. The AC6 orbit is in the dawn-dusk magnetic local times (MLTs) and Fig. 1b shows the number of good 10 Hz samples taken simultaneously by AC6 as a function and L and MLT. Good samples are samples which have a data quality flag of 0. More detailed technical information regarding the AC6 mission can be found in the cite AC6 README.

\begin{figure}
\includegraphics[width=\textwidth]{fig1.png}
\caption{test} \label{fig1}
\end{figure}

\section{Methodology}
\subsection{Microburst Detection}
The first step to find microbursts observed simultaneously by both spacecraft is to identify them from each spacecraft separately. We have detected microbursts with two different methods that yielded quantitatively similar results. The first method is the burst parameter cite Paul’s paper and add equation. This algorithm has been successfully used in other microburst studies, mainly with the microbursts observed by the Solar Anomalous and Magnetospheric Particle Explorer add citations. For AC6, we found that a burst parameter threshold of 5 has good tradeoff between false positive and false negative microburst detections.

Maybe not go into as much in detail in the following paragraph? The other microburst detection algorithm that was developed for this study is based on wavelet transforms and frequency filtering cite Torrence and Compo. The AC6 time series if first transformed into wavelet space by convolving it with a set of Ricker wavelets (more commonly known as the Mexican hat wavelet). An example of the wavelet transformation is shown in Fig. 2. Figure 2a shows the original time series in blue for one radiation belt pass and Fig. 2b shows the wavelet power as a function of period of oscillation and time. At times when microbursts were observed, there is substantial wavelet power in periods less than one second.

A high pass filter at one second was then applied on the wavelet space representation of the microburst time series. Then the remaining wavelet space was inverse filtered to produce a time series which is zero or near-zero everywhere except microbursts. Lastly, a threshold test was applied to identify microbursts. Example detections of microbursts are shown with green stars in Fig. 2a.

\subsection{Transmitter Noise Removal}
The transmitters on AC6 can cause unphysical count impulses in the dosimeters. One source of transmitter noise was observed at times when AC6 was in contact with the ground stations above mainland US for data downloads and commanding. This source of noise mainly manifests itself at lower radiation belt L shells. To account for this noise, detections made above the US were discarded.

Another source of noise is crosslink transmissions between the AC6 units. These transmissions occurred when either spacecraft transitioned from the survey mode to 10 Hz mode. This noise is often not caught by the data quality flag, so an automated noise identification process was developed. To identify this noise, a dosimeter with a 250 keV nominal electron threshold, dos2 was used. Dos2 typically has negligible count rates when dos1 is observing microbursts, and substantial count rates during downlinks and crosslinks. Furthermore, the crosslink transmissions are relatively easy to identify since they are observed near the start and end of the 10 Hz data periods, and are very periodic. The automated noise identification algorithm applied cross-correlation (CC) and autocorrelation (AC) to the dos1 and dos2 time series. Microburst detections were removed when the following two conditions were met. The first condition is true if the dos1 or dos2 time series had an AC peak at 0.2 or 0.4 s lag. The second condition is true if dos2 observed unphysically high count rates or dos1 and dos2 had a Pearson CC coefficient > 0.9. The first condition can be met with a train of microbursts alone and to not remove these valid detections, we imply a second constraint that dos2 experiences unphysical counts or dos1 and dos2 well cross correlate which is unlikely due to an order of magnitude difference in dos1 and dos2 energy thresholds. This admittedly complex algorithm successfully removed most transmitter noise while preserving most valid microburst detections.

\subsection{Coincident Microburst Detection}
At this stage we have lists of microbursts observed by both spacecraft individually and now we combine these lists to identify microbursts observed simultaneously by both AC6 units. Show the microburst detection schematic? The general approach is to CC the time series around microbursts detections made by one spacecraft against the other spacecraft. Ideally, if both spacecraft observed the same microburst, the two time series should correlate well and correlate poorly when a microburst is correlated against random non-microburst times. A CC threshold of 0.8 was chosen as it is a good compromise to identify microbursts superposed with noise and rejecting moderate correlations between a microburst and random features in the other time series.  Due to noise, this CC threshold sometimes failed to reject times when microbursts and non-microburst features were well correlated so all of the events were spot checked by two authors to remove these events. Figure 3, panels (A), (CC), and (E) show examples of microbursts observed by both AC6 units when they were separated by 6, 17, and 69 km.

\begin{figure}
\includegraphics[width=\textwidth]{fig2.png}
\caption{test} \label{fig2}
\end{figure}
	
A physical phenomena that can influence our results are narrow spatial structures termed curtains cite Bern and Paul’s paper. These structures appear as microbursts in a time series from a single satellite, but with two satellites you can adjust the time series of one spacecraft by the in-track lag to identify spatial features. Figure 3b, d, and f show the AC6 spatially aligned time series and show that the these three cases were indeed microbursts.  
	
When the two spacecraft were as little as a few kilometers apart it is very difficult to distinguish between temporal features such as microbursts from spatial features such as curtains. Since the prevalence of curtains is independent of the spacecraft separation, this will effectively reduce the number of microbursts observed at very small separations. No attempt has been made to remove this bias.

\subsection{Microburst Size Distribution in LEO and Magnetic Equator}
When AC6 observes a coincident microburst at a separation d, the microburst’s size must be greater than d. This idea is similar to cite Joy et al., paper who investigated the most probable Jovian bow shock and magnetopause standoff distances. Following Joy’s argument, the fraction of coincident microbursts observed above a distance d is microburst cumulative distribution for the following reason.  If P(A) is the cumulative probability that a microburst is larger than d and P(B) is the probability that AC6 is separated by d, then the fraction of microbursts observed at d is the conditional probability P(A|B). Using Bayes’ theorem, 

P(A|B) = P(A\&B)/P(B)

Where P(A\&B) is the joint probability. Since the AC6 separation is independent of microburst size, P(A\&B) = P(A)P(B). Hence

P(A|B) = P(A)P(B)/P(B) = P(A).


The microburst cumulative probability is then calculated with 

f(d) = N(d)N(0)

where N(d) is the number of microbursts observed by AC6 above d and is defined as
					$N(d) =bins > dnbinsSmaxSbin$
where $n_{bins}$ is the number of coincident microbursts detected in that bin. The normalization term $S_max/S_bin$ is a ratio of the number of samples observed in the most sampled bin to the number of samples in the current bin. This normalization factor corrects for AC6’s non-uniform sampling as a function of separation. With this normalization, f(d) can be interpreted as the fraction of microbursts observed above d assuming AC6 sampled evenly in separation.
	The microburst cumulative distribution in LEO is shown by the black curve in Fig. 4a for the entire radiation belt (4 < L < 8) and split up by 1-L-wide bins with the colored curves. The overall trend consists of a sudden cumulative probability drop off, followed by a shoulder up to around 70 km where the cumulative distribution drops to zero. The shaded region around the black curve shows the standard error due to counting statistics. The cumulative distribution trends can be interpreted as 
	
\begin{figure}
\includegraphics[width=\textwidth]{fig3.png}
\caption{test} \label{fig3}
\end{figure}

\begin{figure}
\includegraphics[width=\textwidth]{fig4.png}
\caption{test} \label{fig4}
\end{figure}

\section{Discussion and Conclusions}
\begin{enumerate}
\item Relate the LEO scale sizes to prior work
\item Compare the equatorial scale size to Oleksiy’s and Santolik’s work. Need to get Oleksiy on board here.
\end{enumerate}






%%

%  Numbered lines in equations:
%  To add line numbers to lines in equations,
%  \begin{linenomath*}
%  \begin{equation}
%  \end{equation}
%  \end{linenomath*}



%% Enter Figures and Tables near as possible to where they are first mentioned:
%
% DO NOT USE \psfrag or \subfigure commands.
%
% Figure captions go below the figure.
% Table titles go above tables;  other caption information
%  should be placed in last line of the table, using
% \multicolumn2l{$^a$ This is a table note.}
%
%----------------
% EXAMPLE FIGURE
%
%
% Giving latex a width will help it to scale the figure properly. A simple trick is to use \textwidth. Try this if large figures run off the side of the page.
% \begin{figure}
% \noindent\includegraphics[width=\textwidth]{anothersample.png}
%\caption{caption}
%\label{pngfiguresample}
%\end{figure}
%
%
%
% If you get an error about an unknown bounding box, try specifying the width and height of the figure with the natwidth and natheight options.
% \begin{figure}
% \noindent\includegraphics[natwidth=800px,natheight=600px]{samplefigure.pdf}
%\caption{caption}
%\label{pdffiguresample}
%\end{figure}
%
%
% PDFLatex does not seem to be able to process EPS figures. You may want to try the epstopdf package.
%
%
%
% ---------------
% EXAMPLE TABLE
%
% \begin{table}
% \caption{Time of the Transition Between Phase 1 and Phase 2$^{a}$}
% \centering
% \begin{tabular}{l c}
% \hline
%  Run  & Time (min)  \\
% \hline
%   $l1$  & 260   \\
%   $l2$  & 300   \\
%   $l3$  & 340   \\
%   $h1$  & 270   \\
%   $h2$  & 250   \\
%   $h3$  & 380   \\
%   $r1$  & 370   \\
%   $r2$  & 390   \\
% \hline
% \multicolumn{2}{l}{$^{a}$Footnote text here.}
% \end{tabular}
% \end{table}

%% SIDEWAYS FIGURE and TABLE
% AGU prefers the use of {sidewaystable} over {landscapetable} as it causes fewer problems.
%
% \begin{sidewaysfigure}
% \includegraphics[width=20pc]{figsamp}
% \caption{caption here}
% \label{newfig}
% \end{sidewaysfigure}
%
%  \begin{sidewaystable}
%  \caption{Caption here}
% \label{tab:signif_gap_clos}
%  \begin{tabular}{ccc}
% one&two&three\\
% four&five&six
%  \end{tabular}
%  \end{sidewaystable}

%% If using numbered lines, please surround equations with \begin{linenomath*}...\end{linenomath*}
%\begin{linenomath*}
%\begin{equation}
%y|{f} \sim g(m, \sigma),
%\end{equation}
%\end{linenomath*}

%%% End of body of article

%%%%%%%%%%%%%%%%%%%%%%%%%%%%%%%%
%% Optional Appendix goes here
%
% The \appendix command resets counters and redefines section heads
%
% After typing \appendix
%
%\section{Here Is Appendix Title}
% will show
% A: Here Is Appendix Title
%
%\appendix
%\section{Here is a sample appendix}

%%%%%%%%%%%%%%%%%%%%%%%%%%%%%%%%%%%%%%%%%%%%%%%%%%%%%%%%%%%%%%%%
%
% Optional Glossary, Notation or Acronym section goes here:
%
%%%%%%%%%%%%%%
% Glossary is only allowed in Reviews of Geophysics
%  \begin{glossary}
%  \term{Term}
%   Term Definition here
%  \term{Term}
%   Term Definition here
%  \term{Term}
%   Term Definition here
%  \end{glossary}

%
%%%%%%%%%%%%%%
% Acronyms
%   \begin{acronyms}
%   \acro{Acronym}
%   Definition here
%   \acro{EMOS}
%   Ensemble model output statistics
%   \acro{ECMWF}
%   Centre for Medium-Range Weather Forecasts
%   \end{acronyms}

%
%%%%%%%%%%%%%%
% Notation
%   \begin{notation}
%   \notation{$a+b$} Notation Definition here
%   \notation{$e=mc^2$}
%   Equation in German-born physicist Albert Einstein's theory of special
%  relativity that showed that the increased relativistic mass ($m$) of a
%  body comes from the energy of motion of the body—that is, its kinetic
%  energy ($E$)—divided by the speed of light squared ($c^2$).
%   \end{notation}




%%%%%%%%%%%%%%%%%%%%%%%%%%%%%%%%%%%%%%%%%%%%%%%%%%%%%%%%%%%%%%%%
%
%  ACKNOWLEDGMENTS
%
% The acknowledgments must list:
%
% >>>>	A statement that indicates to the reader where the data
% 	supporting the conclusions can be obtained (for example, in the
% 	references, tables, supporting information, and other databases).
%
% 	All funding sources related to this work from all authors
%
% 	Any real or perceived financial conflicts of interests for any
%	author
%
% 	Other affiliations for any author that may be perceived as
% 	having a conflict of interest with respect to the results of this
% 	paper.
%
%
% It is also the appropriate place to thank colleagues and other contributors.
% AGU does not normally allow dedications.


\acknowledgments
Enter acknowledgments, including your data availability statement, here.


%% ------------------------------------------------------------------------ %%
%% References and Citations

%%%%%%%%%%%%%%%%%%%%%%%%%%%%%%%%%%%%%%%%%%%%%%%
%
% \bibliography{<name of your .bib file>} don't specify the file extension
%
% don't specify bibliographystyle
%%%%%%%%%%%%%%%%%%%%%%%%%%%%%%%%%%%%%%%%%%%%%%%

\bibliography{/home/mike/Dropbox/0_firebird_research/A_presentations/refs}

%Reference citation instructions and examples:
%
% Please use ONLY \cite and \citeA for reference citations.
% \cite for parenthetical references
% ...as shown in recent studies (Simpson et al., 2019)
% \citeA for in-text citations
% ...Simpson et al. (2019) have shown...
%
%
%...as shown by \citeA{jskilby}.
%...as shown by \citeA{lewin76}, \citeA{carson86}, \citeA{bartoldy02}, and \citeA{rinaldi03}.
%...has been shown \cite{jskilbye}.
%...has been shown \cite{lewin76,carson86,bartoldy02,rinaldi03}.
%...has been shown \cite [e.g.,][]{lewin76,carson86,bartoldy02,rinaldi03}.
%
% DO NOT use other cite commands (e.g., \citet, \citep, \citeyear, \nocite, \citealp, etc.).
%



\end{document}



More Information and Advice:

%% ------------------------------------------------------------------------ %%
%
%  SECTION HEADS
%
%% ------------------------------------------------------------------------ %%

% Capitalize the first letter of each word (except for
% prepositions, conjunctions, and articles that are
% three or fewer letters).

% AGU follows standard outline style; therefore, there cannot be a section 1 without
% a section 2, or a section 2.3.1 without a section 2.3.2.
% Please make sure your section numbers are balanced.
% ---------------
% Level 1 head
%
% Use the \section{} command to identify level 1 heads;
% type the appropriate head wording between the curly
% brackets, as shown below.
%
%An example:
%\section{Level 1 Head: Introduction}
%
% ---------------
% Level 2 head
%
% Use the \subsection{} command to identify level 2 heads.
%An example:
%\subsection{Level 2 Head}
%
% ---------------
% Level 3 head
%
% Use the \subsubsection{} command to identify level 3 heads
%An example:
%\subsubsection{Level 3 Head}
%
%---------------
% Level 4 head
%
% Use the \subsubsubsection{} command to identify level 3 heads
% An example:
%\subsubsubsection{Level 4 Head} An example.
%
%% ------------------------------------------------------------------------ %%
%
%  IN-TEXT LISTS
%
%% ------------------------------------------------------------------------ %%
%
% Do not use bulleted lists; enumerated lists are okay.
% \begin{enumerate}
% \item
% \item
% \item
% \end{enumerate}
%
%% ------------------------------------------------------------------------ %%
%
%  EQUATIONS
%
%% ------------------------------------------------------------------------ %%

% Single-line equations are centered.
% Equation arrays will appear left-aligned.

Math coded inside display math mode \[ ...\]
 will not be numbered, e.g.,:
 \[ x^2=y^2 + z^2\]

 Math coded inside \begin{equation} and \end{equation} will
 be automatically numbered, e.g.,:
 \begin{equation}
 x^2=y^2 + z^2
 \end{equation}


% To create multiline equations, use the
% \begin{eqnarray} and \end{eqnarray} environment
% as demonstrated below.
\begin{eqnarray}
  x_{1} & = & (x - x_{0}) \cos \Theta \nonumber \\
        && + (y - y_{0}) \sin \Theta  \nonumber \\
  y_{1} & = & -(x - x_{0}) \sin \Theta \nonumber \\
        && + (y - y_{0}) \cos \Theta.
\end{eqnarray}

%If you don't want an equation number, use the star form:
%\begin{eqnarray*}...\end{eqnarray*}

% Break each line at a sign of operation
% (+, -, etc.) if possible, with the sign of operation
% on the new line.

% Indent second and subsequent lines to align with
% the first character following the equal sign on the
% first line.

% Use an \hspace{} command to insert horizontal space
% into your equation if necessary. Place an appropriate
% unit of measure between the curly braces, e.g.
% \hspace{1in}; you may have to experiment to achieve
% the correct amount of space.


%% ------------------------------------------------------------------------ %%
%
%  EQUATION NUMBERING: COUNTER
%
%% ------------------------------------------------------------------------ %%

% You may change equation numbering by resetting
% the equation counter or by explicitly numbering
% an equation.

% To explicitly number an equation, type \eqnum{}
% (with the desired number between the brackets)
% after the \begin{equation} or \begin{eqnarray}
% command.  The \eqnum{} command will affect only
% the equation it appears with; LaTeX will number
% any equations appearing later in the manuscript
% according to the equation counter.
%

% If you have a multiline equation that needs only
% one equation number, use a \nonumber command in
% front of the double backslashes (\\) as shown in
% the multiline equation above.

% If you are using line numbers, remember to surround
% equations with \begin{linenomath*}...\end{linenomath*}

%  To add line numbers to lines in equations:
%  \begin{linenomath*}
%  \begin{equation}
%  \end{equation}
%  \end{linenomath*}



